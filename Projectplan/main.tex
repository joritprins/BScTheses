%-------------------------------------------------------------------------------
% LATEX TEMPLATE ARTIKEL
%-------------------------------------------------------------------------------
% Dit template is voor gebruik door studenten van de de bacheloropleiding 
% Informatica van de Universiteit van Amsterdam.
% Voor informatie over schrijfvaardigheden, zie 
%                               https://practicumav.nl/schrijven/index.html
%
%-------------------------------------------------------------------------------
%	PACKAGES EN DOCUMENT CONFIGURATIE
%-------------------------------------------------------------------------------

\documentclass{uva-inf-article}
\usepackage[dutch]{babel}

\usepackage[style=authoryear-comp]{biblatex}
\addbibresource{references.bib}

%-------------------------------------------------------------------------------
%	GEGEVENS VOOR IN DE TITEL, HEADER EN FOOTER
%-------------------------------------------------------------------------------

% Geef je artikel een logische titel die de inhoud dekt.
\title{Policying Android app permissions with eFLINT}

% Vul de naam van de opdracht in zoals gegeven door de docent en het type 
% opdracht, bijvoorbeeld 'technisch rapport' of 'essay'.
\assignment{Projectplan}
\assignmenttype{}

% Vul de volledige namen van alle auteurs in en de corresponderende UvAnetID's.
\authors{Jorit Prins}
\uvanetids{12862789}

% Vul de naam van je tutor, begeleider (mentor), of docent / vakcoördinator in.
% Vermeld in ieder geval de naam van diegene die het artikel nakijkt!
\supervisors{Marco Brohet\\ L. Thomas van Binsbergen}
% \mentor{Marco Brohet}
\docent{}

% Vul hier de naam van je tutorgroep, werkgroep, of practicumgroep in.
\group{}

% Vul de naam van de cursus in en de cursuscode, te vinden op o.a. DataNose.
\course{Afstudeerproject}
\courseid{}

% Dit is de datum die op het document komt te staan. Standaard is dat vandaag.
\date{\today}

%-------------------------------------------------------------------------------
%	VOORPAGINA 
%-------------------------------------------------------------------------------

\begin{document}
\maketitle

%-------------------------------------------------------------------------------
%	INHOUDSOPGAVE EN ABSTRACT
%-------------------------------------------------------------------------------
% Niet toevoegen bij een kort artikel, zeg minder dan 10 pagina's!

%TC:ignore
%\tableofcontents
%\begin{abstract}
%\end{abstract}
%TC:endignore

%-------------------------------------------------------------------------------
%	INHOUD
%-------------------------------------------------------------------------------
% Hanteer bij benadering IMRAD: Introduction, Method, Results, Discussion.

Het projectplan
Al in de eerste week van het afstudeerproject moet een projectplan worden ingeleverd. Een
projectplan is een heldere uitwerking van het hele afstudeerproject en beschrijft het plan van
aanpak. Het projectplan wordt ingeleverd via Canvas nadat het is goedgekeurd door de
projectbegeleider(s) (beide in het geval van een extern project). Het beslaat gemiddeld vier
A4tjes en bestaat uit de volgende onderdelen:
\section{titel}
Uiteraard mogen je naam, titel van het project en de naam van je begeleider(s) niet
ontbreken.
\section{context}
beschrijf hier het onderdeel van het vakgebied waarbinnen dit onderzoek
plaatsvindt. Uit dit deel moet duidelijk zijn hoe de onderzoeksvraag, die later wordt
beschreven, is gepositioneerd binnen de informatica.
De relevante literatuur: wat is de relevante literatuur behorende bij de
onderzoeksvraag? Vaak heeft dit de vorm van een beschrijving van de “state-of-the-art”:
een opsomming van resultaten die eerder door anderen zijn behaald en waar de
onderzoeksvraag op voortbouwt.
\section{De onderzoeksvraag}
beschrijf het probleem waaraan zal worden gewerkt. Vaak heeft
dit de vorm van een reflectie ten opzichte van de zojuist beschreven “state-of-the-art”.
Als onderdeel van de onderzoeksvraag wordt tevens beschreven wat het project gaat
opleveren: het product dat aan het eind zal worden opgeleverd, bv. de resultaten van
een onderzoek, de sourcecode van ontwikkelde software, documentatie.
\section{Methoden}: op welke manier wordt het onderzoek uitgevoerd? Welke subtaken kunnen
worden onderscheiden? Denk aan: uitgebreid literatuuronderzoek, ontwerp en
implementatie van een programma, opzet van een experiment.
\section{De planning} beschrijf hoe de beschikbare tijd naar verwachting zal worden besteed.
Het is vaak lastig om vooraf alle activiteiten te identificeren, laat staan om daarvan op de
dag nauwkeurig aan te geven wat op welke datum af zal zijn. Over het algemeen is het
wel mogelijk een aantal fasen (bv. literatuuronderzoek, ontwerp, implementatie,
experimenten, scriptie schrijven, etc.) te identificeren en daarover een planning op
weekbasis te maken. Denk ook aan de tussentijdse resultaten, de afhankelijkheden die
mogelijk bestaan en of er kritische afhankelijkheden zijn waardoor het project niet door
kan gaan en wat in dat geval het alternatieve plan is. Vaak wordt hiervoor een Gantt- of
Pert-chart gebruikt.
Bijgevoegd is een Excel sheet met de planning.


% \subsectionauthor[Voornaam Achternaam]{Paragraaf met auteur}
% \lipsum[2-3]

%-------------------------------------------------------------------------------
%	REFERENTIES
%-------------------------------------------------------------------------------

\printbibliography

%-------------------------------------------------------------------------------
\end{document}