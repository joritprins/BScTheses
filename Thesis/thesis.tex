\documentclass[twoside]{Style/uva-inf-bachelor-thesis}
\usepackage[dutch]{babel}

\usepackage{graphicx}
\graphicspath{{Afbeeldingen/}}
\usepackage{subfiles}

\usepackage[
    maxbibnames=10,
    style=authoryear-comp
]{biblatex}
\addbibresource{references.bib}

% Title Page
\title{Title}
\author{Author}
\supervisors{Supervisors}
\signedby{Signees}

\begin{document}
\maketitle

\begin{abstract}
Abstract
\end{abstract}

\tableofcontents

\chapter{Introduction}
\section{Introduction}
\begin{itemize}
    \item Start with project proposal
    \item Expand when more information is found for other sections
\end{itemize}
\section{Research questions (from project proposal}
\section{Methods}
% \subfile{Sections/template}
% \subfile{Sections/introductie}

\chapter{Theoretical background}
\section{Overview of field and works}
\subsection{Neural Networks}
\subsection{Secure Neural Networks}
\subsection{General overview of literature on SNNI}
For example: Mann and
\href{https://ieeexplore.ieee.org/stamp/stamp.jsp?tp=&arnumber=9194237}{Tanuwidjaja} gives an chronological overview of works published
\section{Chosen implementation(s)}
\subsection{Cheetah}
Cheetah. Why cheetah?:
+ last commit made on juli 2 and paper from 2022
+ also implementation of cryptflow2
+ wan and lan tested
+ implementation for server and client
+ better than delphi and CryptFlow2	 
+ some others https://arxiv.org/pdf/2205.03040.pdf have used cheetah as a building block and noticed better performance then aby or delphi ( a semi-honest inference protocol (e.g., Cheetah, DELPHI) into a maliciously ) secure
dive deeper in the paper and give a general overview of the implementation

Because of large computation and communication overhead, those systems have been limited to small datasets (such as MNIST and CIFAR) or simple models (e.g. with a few hundreds of parameters). Recently the system CrypTFlow2 [46] has made considerable improvements, and demonstrate, for the first time, the ability to perform 2PC-NN inference at the scale of ImageNet. Despite their advances, there remains considerable overhead: For instance, using CrypTFlow2, the server and the client might need more than 15 minutes to run and exchange more than 30 gigabytes of messages to perform
one secure inference on ResNet50.
\subsection{Chosen implementation2}
if two implementations are compared, ditto.

\chapter{My work}
\section{testing code}

\chapter{Experiments}
\section{RQa, how do we measure the energy consumption}
\section{Design of experiments}
\section{Explaining testing environment}
\section{Results}
\section{RQb, what are the differences on server and client side and what are the implications}

\chapter{Conclusion}
\section{Conclusion}
\section{Discussion}
\section{Ethics}

\end{document}
