\documentclass[twoside]{Style/uva-inf-bachelor-thesis}
\usepackage[dutch]{babel}

\usepackage{graphicx}
\graphicspath{{Afbeeldingen/}}
\usepackage{subfiles}

\usepackage[
    maxbibnames=10,
    style=authoryear-comp
]{biblatex}
\addbibresource{references.bib}

% Title Page
\title{Title}
\author{Author}
\supervisors{Supervisors}
\signedby{Signees}

\begin{document}
\maketitle

\begin{abstract}
Abstract
\end{abstract}

\tableofcontents

\chapter{Introduction}
\section{Introduction}
\begin{itemize}
    \item Start with project proposal
    \item Expand when more information is found for other sections
\end{itemize}
\section{Research questions (from project proposal}
\section{Methods}
% \subfile{Sections/template}
% \subfile{Sections/introductie}

\chapter{Theoretical background}
\section{Overview of field and works}
\begin{itemize}
    \item (subsection: NN and SNNI) (very) short overview of how NN works and what SNNI are)
    \item (subsection: General overview) general overview of the works in literature, for example Mann and \href{https://ieeexplore.ieee.org/stamp/stamp.jsp?tp=&arnumber=9194237}{Tanuwidjaja} gives an chronological overview of works published
\end{itemize}
\section{Chosen implementations}
\subsection{Chosen implementation1}
dive deeper in the paper and give a general overview of the implementation
\subsection{Chosen implementation2}
if two implementations are compared, ditto.
\section{answering of RQa}

\chapter{My work}
\begin{itemize}
    \item Describe the code and how it works
\end{itemize}

\chapter{Experiments}
\section{Explaining testing environment}
\section{Explaining Variables, e.g. bandwith}
\section{results}
\section{Answering of RQb}

\chapter{Conclusion}
\section{Conclusion}
\section{Discussion}
\section{Ethics}

\end{document}
